\documentclass{beamer}
%\documentclass[trans]{beamer}
%\documentclass[handout]{beamer}

\mode<presentation>
{
	%\usetheme{Malmoe}
	\usetheme[secheader]{Boadilla}
	\setbeamercovered{transparent}
}

\usepackage[english]{babel}
\usepackage[latin1]{inputenc}

\usepackage{times}
\usepackage[T1]{fontenc}
% Or whatever. Note that the encoding and the font should match. If T1
% does not look nice, try deleting the line with the fontenc.

\usepackage{graphicx,pgf,tikz}
\usepackage{multicol}
\usetikzlibrary{fadings}

%\usepackage{latexsym}
%\renewcommand{\baselinestretch}{1.2}

\newcommand{\real}{I\!\!\!R}
\newcommand{\bin}{I\!\!\!B}
%\newcommand{\cA}{{\mathcal{A}}}
%\newcommand{\bA}{\bar{\mathcal{A}}}
\newcommand{\cA}{\mathcal{A}}
\newcommand{\bA}{\bar{A}}
\newcommand{\my}{\Gamma} %{M(\bar{y})}
\newcommand{\lb}{\nu}
\newcommand{\scaleAA}{\omega}
\newcommand{\scalei}{\eta_i}
\newcommand{\recipscalep}{\frac{1}{\eta_p}}

\newcommand{\cG}{\mathcal{G}}

\newcommand{\wt}{\widetilde}
\newcommand{\wto}{\widetilde\omega}
\newcommand{\wh}{\widehat}

\newcommand{\sgn}{{\rm sgn}}
\newcommand{\epc}{\hspace{1pc}}
\newcommand{\diag}{{\rm diag}}
\newcommand{\sol}{{\rm SOL}}
\newcommand{\onebld}{{\mbox{\boldmath $1$}}}
\newcommand{\twobld}{{\mbox{\boldmath $2$}}}
\newcommand{\Abld}{{\mbox{\boldmath $A$}}}
\newcommand{\xbld}{{\mbox{\boldmath $x$}}}
\newcommand{\ubld}{{\mbox{\boldmath $u$}}}
\newcommand{\thalf}{{\textstyle \frac{1}{2}}}
\newcommand{\zbld}{{\mbox{\boldmath $z$}}}
\newcommand{\Bbld}{{\mbox{\boldmath $B$}}}
\newcommand{\Cbld}{{\mbox{\boldmath $C$}}}
\newcommand{\Dbld}{{\mbox{\boldmath $D$}}}
\newcommand{\bigf}{\mathcal{F}}
\newcommand{\bigt}{\mathcal{T}}
\newcommand{\bign}{\mathcal{N}}
\newcommand{\SCGF}{\scriptstyle {\mathrm GF}}
\newcommand{\SCNP}{\scriptstyle {\mathrm NP}}

%\newcommand{\ds}{{\textcolor{red}{\circle*{10}}}}
%\newcommand{\tri}{{\textcolor{blue}{\circle*{10}}}}
%\newcommand{\cs}{{\textcolor{green}{\circle*{10}}}}
%\newcommand{\sps}{{\textcolor{yellow}{\circle*{10}}}}

\newcommand{\bz}{\bar{z}}
\newcommand{\tx}{\tilde{x}}
\newcommand{\ty}{\tilde{y}}
\newcommand{\half}{\frac{1}{2}}

\newcommand{\scwork}{\mbox{\scriptsize work}}
\newcommand{\sclb}{\mbox{\scriptsize lb}}
\newcommand{\scub}{\mbox{\scriptsize ub}}

\newcommand{\tred}{\textcolor{red}}
\newcommand{\tblue}{\textcolor{blue}}
\newcommand{\tmag}{\textcolor{magenta}}
\newcommand{\tgreen}{\textcolor{green}}
\newcommand{\bbb}{\textcolor{blue}}
\newcommand{\grr}{\textcolor{purple}}
\newcommand{\tgrey}{\textcolor{gray!25}}
\newcommand{\tpink}{\textcolor{red!60}}
\newcommand{\tcam}{\textcolor{blue!40}}

%\newcommand{\ds}{\textcolor{red}{\circle*{10}}}
%\newcommand{\tri}{{\textcolor{blue}{\circle*{10}}}}
%\newcommand{\cs}{{\textcolor{green}{\circle*{10}}}}
%\newcommand{\sps}{{\textcolor{yellow}{\circle*{10}}}}

%\textheight 8.8in

%\pagestyle{myheadings}


%\markboth{Draft of \today}{Draft of \today}

\newcommand{\til}{\char '176}
\newcommand{\ds}{{\textcolor{red}{\circle*{10}}}}
\newcommand{\tri}{{\textcolor{blue}{\circle*{10}}}}
\newcommand{\cs}{{\textcolor{green}{\circle*{10}}}}
\newcommand{\sps}{{\textcolor{yellow}{\circle*{10}}}}
\newcommand{\tr}{\mbox{trace}}
\newcommand{\re}{I\!\!\!R}
\def\R{{\re}}

%\newtheorem{theorem}{Theorem}
%\newtheorem{lemma}{Lemma}
\newtheorem{proposition}{Proposition}
%\newtheorem{corollary}{Corollary}
%\newtheorem{definition}{Definition}
%\newtheorem{hypothesis}{Hypothesis}

% Definition of the proof-environment:
%\newenvironment{proof}{{\raggedright\bf Proof:}\quad}{\hspace*{\fill}$\fbox{ }$\\}

% Delete this, if you do not want the table of contents to pop up at
% the beginning of each subsection:
%\AtBeginSubsection[]
%{
%  \begin{frame}<beamer>
%    \frametitle{Outline}
%    \tableofcontents[currentsection,currentsubsection]
%  \end{frame}
%}
\AtBeginSection[]
{
	\begin{frame}<beamer>
		\frametitle{Outline}
		\tableofcontents[currentsection,currentsubsection]
	\end{frame}
}

% If you wish to uncover everything in a step-wise fashion, uncomment
% the following command: 

\beamerdefaultoverlayspecification{<+->}

\title[Robust Scheduling]{A Robust Approach for Project Scheduling Problem}
\author[Shen, Yang]{Xin Shen, Jubiao Yang\\{\small Team advisor: John E. Mitchell}}

\institute[RPI]{ 
Rensselaer Polytechnic Institute  \\
Troy,  NY  12180}

\date[MOPTA 2015]{7th AIMMS-MOPTA Optimization Modeling Competition\\\textit{Lehigh University, Bethlehem, PA}, 2015}

\begin{document}

\bibliographystyle{plain}

\begin{frame}

\titlepage

\end{frame}

%\nocite{basescu2,JEM_selorth,basescu3}

\section*{Outline} 
\begin{frame}[allowframebreaks]
\tableofcontents 
\end{frame}

%\section{Abstract}
%\begin{frame}
%\frametitle{Abstract}
%Mathematical programs with complementarity constraints (MPCCs) are
%nonconvex optimization problems with applications in areas such as
%bilevel programming, Stackelberg games, portfolio optimization,
%compressed sensing, and inverse optimization. We discuss progress
%in the development of methods for determining globally optimal
%solutions to some classes of MPCCs. These methods use a variety
%of techniques including logical Benders decomposition, branch-and-cut,
%semidefinite programming, and copositive programming.
%\end{frame}


\section{Introduction}

\begin{frame}
\frametitle{Problem Description}
%
%Want to solve \tblue{large structured optimization problems}.
%
%\bigskip
%
%Exploit the structure:
%
%\medskip
%
%Construct \tblue{sequences of relaxations and/or subproblems},
%
%\medskip
%
%work towards a \tgreen{provably global optimal solution} for the original problem.
\end{frame}
%







\begin{frame}
\frametitle{Eg: Minimal System Realization}
Want to design a Low order Linear Time-Invariant(LTI) system\\
\medskip
Denote by $H_n$ the Hankel matrix with parameters $h_1,h_2,\cdots,h_{2n-1}\in\re$:
\begin{center}
\begin{equation*}
H_n\,=\,\begin{bmatrix}
h_1&h_2&h_3&\cdots&h_n\\
h_2&h_3&h_4&\cdots&h_{n+1}\\
h_3&h_4&h_5&\cdots&h_{n+2}\\
\vdots&\vdots&\vdots&\ddots&\vdots\\
h_n&h_{n+1}&h_{n+2}&\cdots&h_{2n-1}\\\end{bmatrix}
\end{equation*}
\end{center}
%\begin{center}
%\fbox{$\begin{array}{ll}
%\displaystyle{
%{\operatornamewithlimits{\mbox{minimize}}_{(w,b)}}
%} & \underbrace{C \, \displaystyle{
%\sum_{i=1}^n
%} \, \max\left\{ \, | \, w^Tx^i + b - y_i \, |
%- \varepsilon, \, 0 \, \right\} + \thalf \, w^Tw}_{\mbox{standard SVM objective}}   \\ [0.65in]
%& + \, \gamma \, \underbrace{\displaystyle{
%\sum_{i=1}^n
%} \, \left\{ \begin{array}{ll}
%1 & \mbox{if $| \, w^Tx^i + b - y_i \, | \, > \, \varepsilon$}
%\\ [0.1in]
%0 & \mbox{if $| \, w^Tx^i + b - y_i \, | \, \leq \, \varepsilon$}
%\end{array} \right.}_{\mbox{number of misclassified points}}
%\end{array}
%$}
%\end{center}
\bigskip
Fact: Let $h_1, h_2, \cdots h_n$ be givien real numbers. Then there exists a minimal Linear Time-Invariant system of order r if and only if
\begin{center}
\begin{displaymath}
r\,=\, \min\limits_{h_{n+1},\cdots,h_{2n-1}\in\re} ~\mbox{rank} (H_n)
\end{displaymath}
\end{center}
(Fazel)
\end{frame}

\begin{frame}
\frametitle{Eg: Minimal System Realization(Continued)}
The problem can be expressed as:
\begin{center}
\fbox{$\begin{array}{ll}
\displaystyle{
{\operatornamewithlimits{\mbox{minimize}}_{H_n, s}}
} & \mbox{rank}~H_n\\
\mbox{subject to} & l_i\,\leq\,s_i\,\leq\,u_i,~~i\,=\,1,\cdots,n\\
&h_{n+1},\cdots,h_{2n-1}\,\in\,\re
\end{array}$}
\end{center}
where: \\
\medskip
$\bullet$ $s_i\,=\,\sum_{k=1}^i h_k$ denote the terms in the step response. \\
\medskip
$\bullet$ $l_i$ lower bound of step response at the $i$th term.\\
\medskip
$\bullet$ $u_i$ upper bound of step response at the $i$th term.\\
\medskip
(Fazel)
\end{frame}
%
%\begin{frame}
%\frametitle{Eg: }
%Given portfolio $x^0$
%in  standard unit simplex $\Delta_n \triangleq \{ x \in \mathbb{R}^n_+ \mid \onebld^T x = 1 \}$,  \\
%consider a portfolio revision problem as follows:
%\begin{center}
%\fbox{$\begin{array}{ll}
%\displaystyle{
%{\operatornamewithlimits{\mbox{minimize}}_{x \in \Delta_n}}
%} & \underbrace{\displaystyle{
%\frac{c}{2}
%} \, \underbrace{x^TVx}_{\mbox{variance}} + c^{\, \prime} \mbox{ VaR}_{\beta}(x)}_{\mbox{
%composite risk measure}} + \underbrace{K \, \| \, x - x^0 \, \|_0}_{\mbox{portfolio revision cost}}
%\\ [0.65in]
%\mbox{subject to} & \mu^Tx \, \geq \, R \epc \mbox{(expected portfolio return)},
%\end{array}$}
%\end{center}
%where $\mbox{VaR}_{\beta}(x)$
%is the \textcolor{blue}{$\beta$-value-at-risk} of the portfolio $x$ with $\beta \in (0,1)$.
%\end{frame}




\section{Simple Case: Modeling without Bad Luck}

%\subsection*{Motivation}
%
%\begin{frame}
%\frametitle{L0-norm minimization}
%
%Want to solve:
%\begin{displaymath}
%\tblue{\min_x \{f(x) \, + \, \gamma || x ||_0 \, : \, Ax \, \geq \, b, \, x \in \re^n\}}
%\end{displaymath}
%where $A \in \re^{m \times n}$, $b \in \re^m$, and $||x||_0 = \mbox{card}\{x_i \, : \, x_i \neq 0\}$.
%
%\medskip
%
%The parameter $\gamma$ reflects the relative importance of sparsity and~$f(x)$.
%
%%We look at MPCC formulations of this problem and compare them computationally
%%with an $L_1$ convex approximation.
%%
%%\bigskip
%%
%%The $L_1$ convex approximation is:
%%
%%\begin{displaymath}
%%\tmag{\min_x \{ || x ||_1 \, : \, Ax \, \geq \, b, \, x \in \re^n\}}
%%\end{displaymath}
%
%\bigskip
%
%$L_0$-norm minimization is of interest in feature selection,
%compressed sensing, misclassification minimization,
%and almost any problem where a sparse solution is desired.
%
%
%
%\end{frame}


\subsection*{Formulation}


\begin{frame}
\frametitle{SDCMPCC formulation}

Want to solve:
\begin{displaymath}
\tblue{\min_X \{\mbox{rank} (X) \, : \, X\,\in\,\mathcal{C} ~\mbox{and}~ X \in \mathbb{S}^{n}_+\}}
\end{displaymath}

\bigskip

Equivalently:

\begin{displaymath}
\begin{array}{lll}
\min_{X,U} & n\,-\,<I, U> \\  [5pt]
\mbox{subject to} & X\,\in\,\mathcal{C} \\ [5pt]
& 0 \, \preceq \, U \preceq \, I  \\ [5pt]
& {\tblue{0 \, \preceq \, X \,\perp\, U \succeq \, 0}}
%\mbox{subject to} & Ax^+ \, - \, Ax^- \, \geq \, b \\ [5pt]
%& 0 \, \leq \, \xi \leq \, \mathbf{1}  \\ [5pt]
%& \tblue{0 \, \leq \, \xi_i \, \perp \, x^+_i \, + x^-_i \, \geq \, 0} & i=1,\ldots,n  \\ [5pt]
%& \tblue{0 \, \leq \, x^+_i \, \perp \, x^-_i \, \geq \, 0} &  i=1,\ldots,n
\end{array}
\end{displaymath}

\bigskip
When $X$ and $U$ p.s.d, $X\perp U$ is equivalent to:
\begin{displaymath}
<X, U>\,=\,0
\end{displaymath}
\end{frame}

\begin{frame}
Note that if X has the eigenvalue decomposition, 
\begin{displaymath}
X\,=\,P^T \Sigma P
\end{displaymath}
then we can choose 
\begin{displaymath}
U=P_0 \,P_0^T
\end{displaymath}
where $P_0$ is composed of columns in P corresponding to 0 eigenvalue of X.

\bigskip

Thus, it is obvious that $rank(X)=n-<I, U>$.

\end{frame}

\begin{frame}
\frametitle{SDCMPCC formulation}
%If we do not require $X\in\mathbb{S}^n$ to be p.s.d, then we can add constraints:
%\begin{displaymath}
%X\,=\,X^+\,-\,X^-~\mbox{and}~Y\,=\,\left[\begin{array}{clcr}
%X^+&0\\
%0&X^-\end{array}\right]\,\succ\,0
%\end{displaymath}
%In the objective we'll minimize the rank of the p.s.d matrix $Y$.

\bigskip
We can apply the SDCMPCC formulation to the general case $X\in\re^{m\times n}$  by introducing an auxiliary variable Z:
\begin{displaymath}
Z\,=\,\left[\begin{array}{clcr}
G&X^T\\
X&B\end{array}\right]\succeq 0
\end{displaymath} 
For any X, can find matrix $G$ and $B$ such that $Z\succeq 0$ and $rank(Z)=rank(X)$\\
\bigskip
In the objective, we want to minimize the rank of $Z$.

\end{frame}


%No need for any big-$M$ terms in this MPCC formulation.


%\begin{frame}
%\frametitle{Complementarity formulation}
%
%\begin{displaymath}
%\begin{array}{lll}
%\min_{x^{\pm},\xi} & f(x) \, + \, \sum_{i=1}^n (1 - \xi_i)
%\qquad \qquad \qquad \qquad \qquad \qquad \qquad \qquad   \\  [5pt]
%\mbox{subject to} & Ax \, \geq \, b \\ [5pt]
%& 0 \, \leq \, \xi \leq \, \mathbf{1}  \\ [5pt]
%& \only<1>{\tblue{0 \, \leq \, \xi \, \perp \, x}}
%\only<2>{\tmag{0 \, \leq \, \xi \perp \, x^+ + x^- \, \geq \, 0, \,\, 0 \leq x^+ \perp x^- \geq 0, \,\,
%x=x^+-x^-}} 
%%\mbox{subject to} & Ax^+ \, - \, Ax^- \, \geq \, b \\ [5pt]
%%& 0 \, \leq \, \xi \leq \, \mathbf{1}  \\ [5pt]
%%& \tblue{0 \, \leq \, \xi_i \, \perp \, x^+_i \, + x^-_i \, \geq \, 0} & i=1,\ldots,n  \\ [5pt]
%%& \tblue{0 \, \leq \, x^+_i \, \perp \, x^-_i \, \geq \, 0} &  i=1,\ldots,n
%\end{array}
%\end{displaymath}
%
%\bigskip
%
%This is a \tblue{half-complementary} formulation; \\
%\only<1>{\quad}
%\only<2>{can also get a \tmag{full-complementary} formulation by splitting $x=x^+-x^-$.}
%
%\end{frame}

\begin{frame}
\frametitle{Constraint Qualification of SDCMPCC Formulation}
Common Constraint qualifications such as LICQ and Robinson CQ are violated for SDCMPCC.

\bigskip
Here we consider \tblue{Local Calmness}.


\begin{definition}
Suppose that $\bar{x}$ is a local optimal solution to the problem:
\begin{equation}\label{VP}
\displaystyle{
{\operatornamewithlimits{\mbox{minimize}}_{x \, \in \,X}}}  ~~f(x) ~~
\mbox{subject to} ~~ x \in \mathcal{L}~\mbox{and}~~g(x) \in -\mathcal{K}
\end{equation}
 Problem(\ref{VP}) is said to be calm of order $\alpha >0$ at $\bar{x}$ if there exists $M<\infty$ such that, for any sequence $\{z^q\}$ with $0\neq z^q\rightarrow 0$ and any sequence $\{x_q\}\subset \mathcal{L}$ satisfying $x^q\rightarrow \bar{x}$ and $g(x^q)\in z^q -\mathcal{K}$, there holds
\begin{equation}
\frac{f(x^q)\,-\,f(\bar{x})}{||z^q||^{\alpha}}+M \,\geq\,0
\end{equation} 
\end{definition}
%It also holds under  additional assumptions with nonlinear constraints.
%
%\bigskip
\end{frame}

\begin{frame}
\frametitle{Constraint Qualification of SDCMPCC Formulation}
Huang et.al shows that local calmness or order 1 implies the existence of KKT multipliers:
\begin{theorem}
Let $\bar{x}$ be a local optimal solution to Problem(\ref{VP}) and (\ref{VP}) is calm of order 1 at $\bar{x}$. Then, there exists $\mu\in K^*$ such that the system:
\begin{equation*}
\begin{array}{ll}
&0\,\in\,\partial f(\bar{x})\,+\,\mu(\nabla g(\bar{x}))\,+\,N_{\mathcal{L}}(\bar{x})\\[0.15in]
&\mu(g(\bar{x}))\,=\,0
%&\mu_i\,\geq\,0,\qquad\forall i\,=\,1\,\cdots m\\[0.15in]
%&<\mu_{m+1}, g_{m+1}(\bar{x})>\,=\,0\\[0.15in]
%&\mu_{m+1}\,\in\,\mathbb{S}^n_+
\end{array}
\end{equation*}
is consistent. $N_{\mathcal{L}}(\bar{x})$ is the Clarke normal cone of $\mathcal{L}$ at $\bar{x}$.
\end{theorem}

\end{frame}

\begin{frame}
\frametitle{Constraint Qualification of SDCMPCC Formulation}
\begin{proposition}
\tmag{Calmness of Order 1} holds at each local optimum $(\bar{X},\bar{U})$ in the SDCMPCC Formulation.
\end{proposition}
In the proof, let $(X^q, U^q)$ be a feasible solution to the perturbed SDCMPCC Formulation with perturbation parameter $(z^q, r^q, h_1^q, h_2^q)$.\\
\medskip
Want to show the existence of $M<\infty$ that satisfies:
\begin{displaymath}
\tblue{(n - <I, U^q>)\,-\,(n-<I,\bar{U}>)\,\geq\,-M ||(z^q, r^q, h_1^q, h_2^q)||}
\end{displaymath}
for any $(X^q, U^q)\rightarrow(\bar{X},\bar{U})$.\\
\medskip
An upper bound of $n-<I,\bar{U}>$ is $rank(\bar{X})$.
\end{frame}

\begin{frame}
\frametitle{Sketch of Proof}
Want to get a lower bound for $n-<I,U^q>$.
$(X^q, U^q)$ is feasible to the perturbed problem:
\begin{equation*} 
\begin{array}{ll}
\displaystyle{
{\operatornamewithlimits{\mbox{minimize}}_{X, U \, \in \, \mathbb{S}^n}}
} &n\,-\,<I,\,U>\\ [0.15in]
\mbox{subject to} & X\,+\,z^q \in \tilde{\mathcal{C}}\cap \mathcal{S}^n_+\\[0.15in]
&- <X,\,U>\,\leq\,r^q\\[0.15in]
& <X,\,U>\,\leq\,r^q\\[0.15in]
& I\,-\,U\,\succeq\,-h_1^q\, I\\[0.15in]
%& \lambda_{min}(X)\,\geq\, -h_1\\[0.15in]
& U\,\succeq\, -h_2^q\, I
\end{array}
\end{equation*}
The lower bound can be acquired by fixing $X\,=\,X^q$ in the perturbed problem.
\end{frame}

\begin{frame}
\frametitle{Sketch of Proof}
%Want to show the existence of $M<\infty$:
%\begin{displaymath}
%\tblue{(n - <I, U^q>)\,-\,(n-<I,\bar{U}>)\,\geq\,-M ||(z^q, r^q, h_1^q, h_2^q)||}
%\end{displaymath}
%for any $(z^q, r^q,  h_1^q, h_2^q)\rightarrow 0$ and $(X^q, U^q)\rightarrow (\bar{X},\bar{U})$.\\
%\bigskip
%An upper bound of $n-<I,\bar{U}>$ is $rank(\bar{X})$. 
%\bigskip

By fixing $X=X^q$ we can get the following problem:
\begin{equation*} 
\begin{array}{ll}
\displaystyle{
{\operatornamewithlimits{\mbox{minimize}}_{U \, \in \, \mathbb{S}^n}}
} &n\,-\,<I,\,U>\\ [0.15in]
\mbox{subject to} %& X\,+\,z^q \in \tilde{\mathcal{C}}\cap \mathcal{S}^n_+\\[0.15in]
&- <X^q,\,U>\,\leq\,r^q,\qquad y_1\\[0.15in]
& <X^q,\,U>\,\leq\,r^q,\qquad y_2\\[0.15in]
& I\,-\,U\,\succeq\,-h_1^q\, I,\qquad \Omega_1\\[0.15in]
%& \lambda_{min}(X)\,\geq\, -h_1\\[0.15in]
& U\,\succeq\, -h_2^q\, I,\qquad \Omega_2
\end{array}
\end{equation*}
where $y_1, y_2, \Omega_1, \Omega_2$ are the Lagrangian multipliers for the corresponding constraints.
\medskip

$U^q$ is feasible to the above problem.

\medskip
Slater condion holds for the above problem. Can find a lower bound the objective by \tblue{Strong Duality}
\end{frame}

\begin{frame}
\frametitle{Sketch of Proof}
The dual problem is:
\begin{equation*} 
\begin{array}{ll}
\displaystyle{
{\operatornamewithlimits{\mbox{maximize}}_{y_1,y_2\in \mathbb{R},\,\Omega_1, \Omega_2 \, \in \, \mathbb{S}^n}}
} &n+r^q\,y_1+r^q\,y_2-(1+h_1^q)trace(\Omega_1)-h_2^q\,trace(\Omega_2)\\ [0.15in]
\mbox{subject to}% & X_n\,+\,p \in \tilde{\mathcal{C}}\\[0.15in]
& -y_1\,X^q\,+\,y_2\,X^q\,-\,\Omega_1\,+\,\Omega_2\,=\,-I\\[0.15in]
&y_1,\,y_2\,\leq\,0\\[0.15in]
&\Omega_1,\,\Omega_2\,\succeq\,0
\end{array}
\end{equation*}
%\end{frame}
By diagonalizing $X^q$ we can get a tightened problem:
\begin{equation*} \label{TXFixedPSDPerturbedDual}
\begin{array}{ll}
\displaystyle{
{\operatornamewithlimits{\mbox{maximize}}_{y_1,y_2\in \mathbb{R},\,f,\,g \, \in \, \mathbb{R}^n}}
} &n\,+\,r^q\,y_1\,+\,r^q\,y_2\,-\,(1\,+\,h_1^q)\,\sum_i f_i \,-\,h_2^q\,\sum_i g_i\\ [0.15in]
\mbox{subject to}% & X_n\,+\,p \in \tilde{\mathcal{C}}\\[0.15in]
& -y_1\,\lambda_i^q\,+\,y_2\,\lambda_i^q\,-\,f_i\,+\,g_i\,=\,-1,\,\forall i\,=\,1\cdots n\\[0.15in]
&y_1,\,y_2\,\leq\,0\\[0.15in]
&f_i,\,g_i\,\geq\,0,\,\forall i\,=\,1\cdots n
\end{array}
\end{equation*}
%where $\lambda_i^q$ is the $i$th largest eigenvalue of $X^q$.
\end{frame}
%\begin{frame}
%\frametitle{Constraint Qualification of SDCMPCC Formulation}
%
%\begin{quote}
%This point is a \tblue{local minimizer} and a \tmag{strongly stationary point}
%of the SDCMPCC formulation.
%\end{quote}
%When $f(x)\equiv 0$, all feasible x with corresponding assignment of $\xi$ are local minimizers.
%
%\end{frame}
\begin{frame}
\frametitle{Sketch of Proof}
Since $X^q\rightarrow \bar{X}$, $\lambda_i^q\rightarrow\lambda_i$. Can get a lower bound for the objective of the dual problem, which is:
\begin{displaymath}
%\begin{array}
\mbox{rank}(X)\,-\,\frac{2r^q}{\tilde{\lambda}}\,-\,(n-rank(X))(h_1^q+(1\,+\,h_1^q)\frac{2}{\tilde{\lambda}}\,||z^q||)-\,\frac{h_2^q}{\tilde{\lambda}}\,||\bar{X}||^*
%\end{array}
\end{displaymath}
where $\tilde{\lambda}$ is the smallest positive eigenvalue of $\bar{X}$.
\bigskip
We can take 
\begin{displaymath}
M\,=\,\frac{2}{\tilde{\lambda}}\,+\,\frac{1}{\tilde{\lambda}}||\bar{X}||^*\,+\,(n-rank(X))(1+\frac{4}{\tilde{\lambda}})
\end{displaymath}
\end{frame}

\begin{frame}
\frametitle{KKT Condition of SDCMPCC Formulation}
Given $\mathcal{C}\,=\,\{X\,|\,<A_i, X>\, \geq \, b_i, \, \forall i\,=\,1\cdots p\}$\\
The KKT condition is:
\begin{equation}\label{ExactKKTCondition}
\begin{array}{ll}
0\,\preceq\,U&\,\perp\,-I\,+\,\mu\, X\,+Y\,\succeq 0 \\[0.15in]
0\,\preceq\,X&\,\perp\,-\sum\lambda_i A_i\,+\,\mu\, U\,\succeq 0 \\[0.15in]
0\,\preceq\,Y&\,\perp\,I- U\,\succeq 0\\[0.15in]
0\,\,\leq\,\lambda_i&\,\perp\,b_i\,-\,<A_i, X_i>\,\geq\,0, \forall i\,=1\,\cdots,p
\end{array}
\end{equation}
Where $\lambda, \mu$ and Y are lagrangian multipliers corresponding to the constraints $A(X)=b$, $<X,U>=0$ and $I-U\succeq 0$ respectively.\\
\bigskip
\tblue{Any feasible pair $(X, U)$ with U given by $P_0 P_0^T$ with columns of $P_0$ to be the eigenvectors in the null space of $X$ ,is a KKT stationary point of the SDCMPCC Formulation.}

%\bigskip
%
%Under certain conditions, the solution to the relaxed formulation is a
%solution to a reweighted version of the $L1$-norm minimization problem.

\end{frame}







%\begin{frame}
%\frametitle{Comparison with Weighted $L1$ Optimization Problem}
%Definition: A point y is called \tmag{nondominated} if there does not exist $\tilde{y}$ such that $|\,\tilde{y}\,|\,\leq\, |\,y\,|,\tilde{y}\,\neq\, \,y$
%
%\bigskip
%Nondominated points in the feasible region can be obtained by Weighted $L1$ Minimization with strict positive weights.
%
%\bigskip
%The limit point $x^*$ of local optimums to $NLP(\epsilon)$ must be nondominated.
%\end{frame}
%
%\begin{frame}
%%Give ExpLP as an example
%%\frametitle{$x=()$ is not a local optimum to NLP }
%\tmag{Not all} nondominated points are the limit point of local optima of $NLP(\epsilon)$  as $\epsilon\rightarrow 0$. 
%Consider the following problem:\\
%%\begin{minipage}{2.1in}
%%\includegraphics[scale=0.25]{ExpLP.jpg}
%%\end{minipage}
%\begin{minipage}{0.1in}
%\
%\end{minipage}
%\begin{minipage}{2.1in}
%\begin{equation*}\label{ExampleL0}
%\begin{array}{ll}
%\displaystyle{
%{\operatornamewithlimits{\mbox{minimize}}_{x \, \in \, \mathbb{R}^2}}
%} & \| \, x \, \|_0 \\ [0.15in]
%\mbox{subject to} & 10\, x_1\,+\,x_2 \, \geq \,21 \\
%&x_1\,+\,2\,x_2\,\geq\,4\\
%&x \, \geq \, 0
%\end{array}
%\end{equation*}
%\end{minipage}
%\bigskip\\
%(2,1) is a solution to weighted $L1$ minimization problem for certain choice of weight.
%
%\bigskip
%However, it can not be the x part of a limit point of local optima of $NLP(\epsilon)$ as $\epsilon\rightarrow 0$. 
%\end{frame}
%
%%\begin{frame}
%%\frametitle{Relationship with a Piecewise Problem}
%%Want to solve a generalized version of $L0$ optimization problem:
%%\begin{displaymath}
%%\tblue{\min_x \{f(x) \, + \, \beta || x ||_0 \, : \, c_{\mathcal{E}}(x) \, = \, 0, c_{\mathcal{I}}(x) \, \leq \, 0, \, x \in \re^n\}}
%%\end{displaymath}
%%\bigskip
%%
%%For any feasible point $\bar{x}$, consider a piecewise problem
%%\begin{equation*}
%%\tblue{\min_x \{f(x) : \, c_{\mathcal{E}}(x) \, = \, 0, c_{\mathcal{I}}(x) \, \leq \, 0, \, x_i=0\, \textrm{ if } \bar{x}_i=0,\,x\in \re^n\}}
%%\end{equation*}
%%\bigskip
%%
%%Under LICQ, if $\bar{x}$ is the limit point of local optimums to $NLP(\epsilon)$ as $\epsilon\rightarrow 0$, then  Second Order Necessary Condition is satisfied at $\bar{x}$ in the piecewise problem.
%%\end{frame}
%
%\begin{frame}
%\frametitle{Completely positive formulation}
%The cone of completely positive matrices is defined as:
%\begin{equation*}
%\mathcal{C}_n\,=\,\{\sum_{i=1}^k x_i x_i^T: x_i\in \mathcal{R}_+^n\mbox{ for } i\,=\,1\cdots k\}
%\end{equation*}
%
%\bigskip
%< , > denotes the Frobenius product of two matrices
%\end{frame}
%
%\begin{frame}
%\frametitle{Completely positive formulation(Continue)}
%Consider the problem with $L0$ objective, and the feasible region is the intersection of a linear subspace and nonnegative orthant.
%\begin{displaymath}
%\begin{array}{lll}
%\min_{x,\xi} & \sum_{i=1}^n (1 - \xi_i)  \\  [5pt]
%\mbox{subject to} &  Ax \, = \, b, \\ [5pt]
%%& 0 \, \leq \, x^+, x^- \\ [5pt]
%& 0 \, \leq \,\xi \, \leq \, \mathbf{1}\\ [5pt] 
%& 0\, \leq\,x \\[5pt]
%%&\xi\, \leq \, \mathbf{1}\\ [5pt]
%& \sum_{i=1}^n\xi_i \,x_i\,\leq\,0 \\ [5pt]
%%&\xi_i \, x_i \,\leq\,\epsilon,\qquad \mu^-_i & i=1,\ldots,n % \\ [5pt]
%%& \tblue{0 \, \leq \, x^+_i \, \perp \, x^-_i \, \geq \, 0} &  i=1,\ldots,n
%\end{array}
%\end{displaymath}
%Add an auxillary variable $s\geq 0$ and replace the contraint $\xi\leq\mathbf{1}$ by $\xi + s = \mathbf{1}$
%\end{frame}
%
%
%\begin{frame}
%\frametitle{Completely positive formulation(Continue)}
%Have the completely positive formulation:
%\begin{equation*} \label{eq:COP}
%\begin{array}{ll}
%\displaystyle{
%{\operatornamewithlimits{\mbox{minimize}}_{x, \, \xi}}
%} & \onebld_n^T ( \, \onebld_n - \xi \, ) \, = \, \displaystyle{
%\sum_{i=1}^n
%} \, ( \, 1 - \xi_i \, ) \\ [0.15in]
%\mbox{subject to}&\left[\begin{array}{clcr}
%1&x^T&\xi^T&s^T\\
%x&Y_1&Y_4^T&Y_6^T\\
%\xi&Y_4&Y_2&Y_5^T\\
%s&Y_6&Y_5&Y_3\end{array}\right]\, \in\, \mathcal{C}_{3n+1}\\
% \mbox{and}& a_i^T x\, = \, b_i,\qquad\forall i\in 1\cdots m,\\ [7pt]
%&<a_i a_i^T\,,\,Y_1>\,=\,b_i^2,\qquad\forall i\in 1\cdots m,\\[7pt]
%%& 0 \, \leq \, \xi \\ [7pt]
%%&0\,\leq\, x\\[7pt]
%& <\mathcal{I}_n,\,Y_4>\,\leq\,0\\ [7pt]
%&\xi_i\,+\,s_i\,=\,1,\qquad\forall i=1\cdots n,\\[7pt]
%& <diag(e_i)\,,\,Y_2+Y_3+2*Y_5>\,=\,1, \forall i=1\cdots n
%\end{array}
%\end{equation*}
%\end{frame}
%
%\begin{frame}
%\frametitle{Completely positive formulation}
%The completely positive formulation has the same objective value with the complementarity formulation. The proof does not require the boundedness of x.\\
%\bigskip
%All the constraints are convex.\\
%\bigskip
%It is NP-hard to verify whether a matrix is completely positive or not
%\end{frame}
%
%
%
%
%
%
%%\begin{frame}
%%\frametitle{Penalty method}
%%Can \tblue{penalize violations of complementarity} in many ways:  \\ [5pt]
%%
%%$\bullet$ Add $(x^++x^-)^T\xi$ to the objective function. \\ [5pt]
%%
%%$\bullet$ Add $(x^++x^-)^T\xi \, + \, (x^+)^Tx^-$  to the objective function.  \\ [2pt]
%%\quad Corresponds to the AMPL keyword ``complements'' \\ \quad when using knitro.  \\ [5pt]
%%
%%{$\bullet$ Add $\tred{(}((x^++x^-)^T\xi\tred{)^2}$ to the objective function.}  \\ [5pt]
%%
%%\only<1>{$\bullet$ Add $\tred{(}(x^++x^-)^T\xi \, + \, (x^+)^Tx^-)\tred{)^2}$  to the objective function.}
%%\only<2>{\tblue{$\bullet$ Add $\tred{(}(x^++x^-)^T\xi \, + \, (x^+)^Tx^-)\tred{)^2}$  to the objective function.}} \\ [5pt]
%%
%%$\bullet$ Use the Fischer-Burmeister function, or some other NCP function.
%%
%%\bigskip
%%
%%Theoretical investigation is ongoing.
%%
%%\bigskip
%%
%%\pause
%%
%%Surprisingly, the \tblue{fourth formulation} has worked well in our tests.
%%
%%
%%\end{frame}
%%
%
%\subsection*{Computational Results}
%
%\begin{frame}
%\frametitle{Test problems}
%Examine three classes of problems:  
%
%\bigskip
%
%$\bullet$ Minimize L0-norm, without another objective~$f(x)$.
%
%\medskip
%
%$\bullet$ Look at weighted combinations of $f(x)$ and L0-norm.
%
%\medskip
%
%$\bullet$ Look at signal recovery problems.
%
%\bigskip
%
%In each case, compare solutions obtained by a nonlinear programming
%package (KNITRO, SNOPT, CONOPT, MINOS) for our complementarity formulation
%with an L1-norm formulation.
%
%\bigskip
%
%Have no guarantee of global optimality.
%
%\end{frame}
%
%\subsection*{Minimize L0-norm}
%
%\begin{frame}
%\frametitle{Minimize number of non zeroes}
%
%
%\begin{displaymath}
%\begin{array}{ll}
%  \min_{x\in\real^n} \quad&  \|x\|_{0,1} \\
%  \text{s.t.} \quad& A x \geq b.
%\end{array}
%\end{displaymath}
%
%Look at four formulations:
%
%\bigskip
%
%\begin{tabular}{ll}
%\tblue{MILP:} & integer programming formulation with a big-$M$ \\  [5pt]
%\tblue{L1:} & $L1$-approximation solved as an LP.  \\  [2pt]
%&  Objective reweighted iteratively (Candes et al). \\  [5pt]
%\tblue{fullcomp:} & complementarity formulation solved as NLP  \\  [5pt]
%\tblue{halfcomp:} & modified complementarity formulation, solved as NLP
%\end{tabular}
%
%\bigskip
%
%
%\tgreen{Test problems}
%
%$A$ and $b$ generated randomly, each entry in $U[-1,1]$.
%\end{frame}
%
%%\begin{frame}
%%\frametitle{$A$ is $30 \times 50$: \\ 50 test problems, various solvers and start points}
%%\includegraphics[scale=0.5]{Test1_Small_LP_Reweight.jpeg}
%%\end{frame}
%
%%\begin{frame}
%%\frametitle{$A$ is $300 \times 500$: 50 test problems}
%%\includegraphics[scale=0.5]{Test1_Large_LP_Reweight.jpeg}
%%\end{frame}
%
%%\begin{frame}
%%\frametitle{$A$ is $300 \times 500$: performance profile of sparsity}
%%\includegraphics[scale=0.5]{Test1_PerformanceProfile.jpeg}
%%\end{frame}
%
%
%%\begin{frame}
%%\frametitle{Comments on preliminary results}
%%
%%The complementarity formulations generally give \tblue{sparser solutions}
%%than the $L1$ LP formulation.
%%
%%\medskip
%%
%% Best of  ten runs with Knitro is \tmag{always better} than  $L1$ solution:  \\  [5pt]
%% \qquad on average, only 26\% worse than MILP solution.  \\ %[3pt]
%% \qquad in contrast, $L1$ solution is 3 times worse than MILP solution.
%%
%%\medskip
%%
%%Occasionally, a complementarity formulation gives a bad solution.
%%This is less true with the last two starting point choices.
%%Both these choices exploit a \tblue{warm start} based on the $L1$ LP solution.
%%
%%\medskip
%%
%%The \tblue{half-complementarity} formulation gives fewer poor solutions than the
%%full-complementarity formulation.
%%
%%%\medskip
%%%
%%%We are trying to understand why Knitro is so effective.
%%
%%\end{frame}
%
%\subsection*{Pareto frontier}
%
%\begin{frame}
%\frametitle{Weighted combinations}
%
%\begin{displaymath}
%\begin{array}{ll}
%  \min_{x\in\real^n} \quad&  q(x) + \beta\|x\|_{0,1} \\
%  \text{s.t.} \quad& A x \geq b.
%\end{array}
%\end{displaymath}
%
%\bigskip
%
%Solve for each norm for various different choices of~$\beta$.
%
%\bigskip
%
%In our tests, $A$ is $30 \times 50$, $q(x)$ is  strictly convex
%quadratic function.
%
%\medskip
%
%The L0-norm formulations are solved using KNITRO under AMPL.
%
%\end{frame}
%
%%\begin{frame}
%%\frametitle{Pareto frontier}
%%\includegraphics[scale=0.5]{Test2_Small_Pareto.jpeg}
%%\end{frame}
%
%\subsection*{Signal recovery}
%
%\begin{frame}
%\frametitle{Signal recovery}
%
%\begin{displaymath}
%  \min_{x\in\real^n} \quad  \|Ax-b\|_2^2 + \beta\|x\|_{0,1}
%\end{displaymath}
%
%\bigskip
%
%Look at different choices of regularizing parameter $\beta$.
%
%\medskip
%
%$A$ is $256 \times 1024$. \\
%In original signal, 40 entries of $x$ are nonzero. \\
%Random noise is added to give~$b$.
%
%\bigskip
%
%L1-norm results are improved by \tblue{debiasing}:  \\
%fix zeroes in L1 solution at zero, then look for best least squares fit.  \\
%L1-debiased leads to many small nonzero components in our tests.
%
%\end{frame}

%\begin{frame}
%\frametitle{Signal recovery mean square error}
%\includegraphics[scale=0.7]{Test3_CPLEXMSE.pdf}
%\end{frame}

%\begin{frame}
%\frametitle{Signal recovery sparsity}
%\includegraphics[scale=0.7]{Test3_CPLEXSparsity.pdf}
%\end{frame}
\section{Robust Scheduling with Bad Luck}


\section{Conclusions}






\end{document}
